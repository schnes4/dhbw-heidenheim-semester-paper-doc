%!TEX root = ../../main.tex

\chapter{Modellhafte Darstellung zur Evaluierung}

\section{Versuchsaufbau}

\subsection{Motorenmodell}

\subsection{Hardwaremodell}
\label{subsec:hardwaremodell}

\section{Technische Umsetzung}

\subsection{Detektion des Propellerblattes}
Um neben einer Unwuchtdetektion auch eine Unwuchtlokalisierung zu ermöglichen, ist die Detektion mindestens eines Propellerblattes zwingend notwendig.
Für die Lokalisierung der Unwucht muss der Arduino die Unwuchtmessung jeweils pro vollständiger Umdrehung ($360^\circ$ Umdrehung) des Propellers starten.
Durch die Detektion beider Propellerblätter ergeben sich prinzipiell Vorteile bei der Zuordnung der Messergebnisse der Unwucht. 
Da sich dies jedoch, wie in den folgenden Kapiteln aufgezeigt wird, aufgrund der technischen Umsetzung als problematisch herausstellt und durch die Detektion eines Propellerblattes und der Umdrehungsgeschwindigkeit des Propellers Rückschlüsse auf das zweite Propellerblatt gezogen werden können, wird im Zuge dieser Studienarbeit auf die Detektion des zweiten Propellerblattes verzichtet.

\subsubsection*{Theoretische Vorgehensweise}
Für die Detektion des Propellerblattes werden zwei technische Lösungen erarbeitet, welche im späteren Verlauf dieses Kapitels gegenüber gestellt werden.
Beiden Lösungen liegt jedoch die gleiche theoretische Vorgehensweise zugrunde:
\begin{enumerate}
	\item Eine Seite des Propellers schwarz anmalen / schwarz bekleben
	\item Propeller mit definierter Drehzahl rotieren lassen
	\item Bei Detektion der schwarzen Fläche Interrupt auslösen
\end{enumerate}


\begin{figure}[h]
	\centering
	\includegraphics[width=.9\textwidth]{chapter/03/propeller_detektion.jpg}
	\caption{Modifikation zur Detektion eines Propellerblattes}
	\label{fig:propeller_detektion}
\end{figure}
Wie in Abbildung \ref{fig:propeller_detektion} erkennbar ist, gestaltet sich die Detektion des zweiten Propellerblattes als schwierig. 
Um das Propellerblatt für den Sensor erkennbar zu machen, muss dies auf der zweiten Seite ebenfalls schwarz bemahlt bzw. beklebt werden.
Dies führt jedoch dazu, dass bei laufendem Motor zwar zwischen beiden Propellerblättern unterschieden werden kann (jedes zweite erkannte Propellerblatt ist das gleiche), jedoch keine verlässliche Aussage möglich ist, an welchem der beiden Propellerblätter die Unwucht letztlich wirklich ist, da eine eindeutige Unterscheidung zwischen den Propellerblättern durch den Sensor nicht möglich ist.

Treten beispielsweise durch Lichteinflüsse bedingte Messfehler auf, kann dies zu einem überspringen bzw. nichtdetektieren des Propellerblattes fürhen.
Wird die Messung mit nur einem bemalten bzw. beklebten Propellerblatt durchgeführt und durch einen Messfehler nicht erkannt, ist die Messung bis zur nächsten Detektion prinzipiell falsch, kann aber später als Ausreißer ignoriert werden.
Werden beide Seiten bemalt bzw. beklebt und eine der beiden Seiten wird übersprungen, verschiebt sich die Position des Propellerblattes für die restliche Messung.
Dies würde zu einem vollkommen falschen Messergebnis führen.
Falsche Werte könnten nicht mehr als Ausreißer aussortiert werden, da diese ab dem Zeitpunkt des Messfehlers bis zum Zeitpunkt des nächsten Messfehlers bzw. bis zum Ende der Messung per se richtig wären, jedoch um 180$^\circ$ verschoben.

\subsubsection*{Vergleich zwischen Sensoren}
Zur Detektion des Propellerblattes stehen zwei verschiedene Arten an Sensoren zur Auswahl, welche im Folgenden gegenübergestellt und bewertet werden.

\newcolumntype{M}{>{\centering\arraybackslash}X}
\begin{table}[h]
	\centering
	\begin{tabularx}{0.95\textwidth}{M|M}

		Lichtsensor & IR-Reflektions-Sensor \\
		\hline
		\includegraphics[height=3cm]{images/chapter/03/sensor_tsl2591.jpg} & \includegraphics[height=4cm]{images/chapter/03/sensor_ir.jpg}  \\
		\includegraphics[height=3cm]{images/chapter/03/sensor_tsl2591_schema.png} & \includegraphics[height=3cm]{images/chapter/03/sensor_ir_schema.png} \\
		Adafruit TSL2591 & Adafruit Reflective IR Sensor \newline \\
	\end{tabularx}
	\caption{Vergleich zwischen zwei verschiedenen Arten an Sensoren zur Detektion des Propellerblattes}
\end{table}

Wie bereits beschrieben, erfolgt die Detektion des Propellerblattes bei beiden Sensoren durch die Detektion einer schwarzen Fläche. 
Allerdings ergeben sich durch die technologischen Unterschiede diverse Aspekte welche eine Einsetzbarkeit im Kontext dieser Studienarbeit beeinflussen.
Neben der Differenzierung zwischen infrarotem Lichtspektrum und sichtbarem Lichtspektrum bei der Messung, weist der Lichtsensor eine sehr hohe Lux Reichweite auf \cite[S.1]{tsl2591_datasheet}.
\begin{figure}[H]
	\centering
	\includegraphics[width=\textwidth]{images/chapter/03/lichtspektrum.png}
	\caption{Darstellung des gesamten Lichtspektrums \cite[]{light_spectrum}}
\end{figure}
Hierduch können unerwünschte Lichteinflüsse, wie beispielsweise sehr helle Sonneinstrahlung, vermieden werden.
Dies erlaubt den Einsatz des Lichtsensors in nahezu allen Lichtverhältnissen - ein nicht zu vernachlässigender Aspekt, da hierdurch die Unwuchtmessung am realen Flugzeugmotor nicht zwingend in einer Umgebung mit kontrollierbaren Lichteinflüssen wie beispielsweise einem Hangar stattfinden muss.
Des Weiteren bietet der Lichtsensor eine integrierte Interrupt Schnittstelle, wodurch die Komplexität deutlich reduziert werden kann im Vergleich zu einem \ac{IR}.
Trotz der Tatsache, dass der Lichtsensor in den soeben beschriebenen Aspekten einem \ac{IR} deutlich überlegen ist, bietet ein \ac{IR} deutliche Vorteile, welche letztlich ausschlaggebend sind für die Entscheidung, welcher der beiden Sensoren verwendet werden soll.
Der wichtigste Aspekt hierbei ist jedoch die deutlich höhere Abtastrate des \ac{IR} im Vergleich zum Lichtsensor. 
\begin{figure}[H]
	\centering
	\includegraphics[width=0.75\textwidth]{images/chapter/03/sensor_ir_schaltung.png}
	\caption{Beispielhafte Verwendung des \ac{IR}s \cite{sensor_ir_schaltung}}
	\label{fig:ir_example}
\end{figure}
Abbildung \ref{fig:ir_example} stellt einen exemplarischen Verwendungszweck des \ac{IR}s dar. 
Der Sensor (in Abbildung \ref{fig:ir_example} durch ein S repräsentiert) ist sowohl direkt mit dem Arduino, als auch mit einem Pull-Up Wiederstand verbunden.
Dies hat zur Folge, dass am Port des Arduinos 5\ac{V} anliegen, solange der Sensor eine Reflexion des vom Emitter (in Abbildung \ref{fig:ir_example} durch ein E repräsentiert) ausgesendeten Lichtes erhält.
Wird das Licht nicht reflektiert - dies ist der Fall sobald sich die schwarze Fläche des Propellerblattes vor dem \ac{IR} befindet - liegt am Arduino Port 0\ac{V} an.
Für die Detektion des Propellerplattes, respektive für den Wechsel zwischen 5V und 0V (wird auch als Rise bzw. Fall-Time bezeichnet) benötigt der \ac{IR} lediglich eine Zeit von 8$\mu$s ausgehend einer Temperatur von 25$^\circ$C \cite[S.2]{ir_datasheet}.
Der Lichtsensor hingegen integriert alle gemessenen Werte über einen zuvor definierten Zeitraum. Das Problem hierbei ist, dass für die Dauer des Zetiraums lediglich folgende Werte verwendet werden können \cite[S.13]{tsl2591_datasheet}:
\begin{itemize}
	\item 100ms
	\item 200ms
	\item 300ms
	\item 400ms
	\item 500ms
	\item 600ms
\end{itemize}
Ein Flugzeugmotor dreht im Leerlauf mit ca. 2500 U/s. Um ein valides Ergebnis zu erzielen, muss die minimale Abtastrate höher als die maximale Umdrehungszahl des Motors sein, um sicherzustellen, dass der schwarz markierte Bereich des Propellerblattes auch detektiert wurde.
Umgerechnet bedeutet dies, dass der Propeller für eine Umdrehung ca. 24ms benötigt.
Da die geringste Abtastrate des Lichtsensors jedoch 100ms beträgt, ist dieser für den Anwendugsfall dieser Studienarbeit ungeeignet.
Aus diesem Grund wurde für die Umsetzung dieser Studienzeit auf den \ac{IR} als Sensor zur Detektion des Propellerblattes gesetzt.

\subsubsection*{Probleme bei der Detektion des Propellerblattes mit einem \ac{IR}}
Aufgrund der technischen Beschaffenheit des \ac{IR}s ist eine Detektion des Propellerblattes in der Theorie ohne weiteres möglich.
Die Praxis zeigt jedoch, dass hierbei einige Probleme auftreten können.

Ein Problem ist die Positionierung des \ac{IR}s.
Dieser benötigt einen Abstand zwischen Sensor und Propellerblatt von minnimal 2mm bis maximal 10mm \cite[Description]{sensor_ir_description}.
Desweiteren ist der passende Winkel zum Propellerblatt für eine erfolgreiche Detektion entscheidend.
Es ist nicht ausreichend, den Sensor fix an einer Stelle zu befestigen, da sich durch die Zentrifugalkräfte der gewichteten Propellerblätter (Gewichte werden verwendet um Unwucht zu erzeugen - hierauf wird in Kapitel \ref{subsec:detektion_der_unwucht} genauer eingegangen) der Abstand zwischen Propellerblatt und Sensor verändert.
\begin{figure}[H]
	\centering
	\includegraphics[width=0.9\textwidth]{images/chapter/03/exp_ir_sensor.jpg}
	\caption{Befestigung des Sensors mithilfe eines Klettbandes}
	\label{fig:exp_ir_sensor}
\end{figure}
Aus diesem Grund wird die Holzkonstruktion mit einem Klettband versehen, wie in Abbildung \ref{fig:exp_ir_sensor} zu sehen ist.
Dies ermöglicht es, den Sensor für jeden Propeller individuell einzustellen.

Neben dem Abstand bzw. dem Winkel des \ac{IR}s spielen Lichteinflüsse eine Rolle.
Im Gegensatz zum Lichtsensor ist der \ac{IR} trotz einer zusätzlichen Schutzklappe anfälliger gegenüber auf den Sensor einströmendes Licht \cite[Features]{ir_datasheet}.
Dies hat zur Folge, dass der Versuch nur in einer Umgebung mit kontrollierten Lichteinflüssen erfolgreich stattfinden kann.
Auffällig ist dieses Phänomen, sobald sich der Sensor direkt vor einem Fenster mit Blickrichtung auf dieses befindet.
Zu geringes Licht ist ein weiteres Problem, da hier die schwarze Fläche des Propellerblattes durch den \ac{IR} nicht mehr zuverlässig detektiert werden kann.

Das größte Problem stellt die Kombination eines \ac{9-DOF} Sensors und \ac{IR}s an einem einzigen Arduino dar.
\begin{figure}[H]
	\centering
	\includegraphics[width=0.9\textwidth]{images/chapter/03/hardware-layout-with-one-arduino.png}
	\caption{Altes Hardwaremodell mit einem Arduino}
\end{figure}
Bei gleichzeitigem Messen beider Sensoren ist eine zuverlässige Detektion des Rotorblattes nicht möglich.
Abbildung \ref{fig:interrupts_are_shit} stellt die Messergebnisse des \ac{IR}s graphisch dar.
\begin{figure}[H]
	\centering
	\includegraphics[width=0.9\textwidth]{images/chapter/03/interrupts_are_shit.png}
	\caption{Messwerte des \ac{IR}s bei gleichzeitiger Messung eines \ac{IR}s und eines \ac{9-DOF}s}
	\label{fig:interrupts_are_shit}
\end{figure}
Wie in Abbildung \ref{fig:interrupts_are_shit} zu erkennen ist, wird das Propellerblatt nur unregelmäßig erkannt.
Des Weiteren werden teilweise 3 Interrupts durch \ac{ISR} erkannt.

Aus diesem Grund erfolgt eine Aufteilung der Sensoren auf die zwei Arduinos \ac{A1} und \ac{A2} wie in Abbildung \todo{ref auf hardwaremodell bild setzen} dargestellt. 
Weiterhin erfolgt eine Aufgabenteilung, wie in Abbildung \ref{fig:aufgaben-arduinos} dargestellt.
\begin{figure}[H]
	\centering
	\includegraphics[width=0.9\textwidth]{images/chapter/03/aufgaben-arduinos.png}
	\caption{Aufgaben der beiden Arduinos \ac{A1} und \ac{A2}}
	\label{fig:aufgaben-arduinos}
\end{figure}
Durch die Trennung beider Sensoren und die Aufgabenteilung in Propellerblattdetektion und Unwuchtmessung ist \ac{A1} nun in der Lage das Propellerblatt konstant (im Rahmen einer gewissen Fehlertoleranz) zu detektieren.
\begin{figure}[H]
	\centering
	\includegraphics[width=0.9\textwidth]{images/chapter/03/self_made_interrupt_a1.png}
	\caption{Messerte des \ac{A1}}
	\label{fig:self_made_interrupt_a1}
\end{figure}

\subsection{Detektion der Unwucht}
\label{subsec:detektion_der_unwucht}

\subsubsection*{Theoretische Vorgehensweise}
Aufgrund der Unwucht des Propellers entstehen Vibrationen am Motorblock
Durch Anbringung eines Sensors direkt am Motorblock können diese Vibrationen gemessen werden
Je nach Auslenkung des Motorblockes in Abhängigkeit von dem aktuell vorbeidrehenden Propeller sollten so Rückschlüsse auf die Herrkunft (also von welchem Propeller aus) der Unwucht getroffen werden können

\subsubsection*{Sensortechnologie}
Verwendeter Sensor: Adafruit LSM9DS1
Eingesetzter Sensor ist ein 9-DOF (\textbf{D}egrees \textbf{f} \textbf{F}reedom). Er ist mit 4 verschiedenen Sensoren ausgestattet. Darunter ein Beschleunigsungssensor welche die Beschleunigung des Sensors in drei verschiedene Richtungen misst, ein Gyroskop welcher die Ausrichtung des Sensors misst und ein Magnetometer welcher die Lage des Sensors in Abhängigkeit des Erdmagnetfeldes darstellen kann. Der vierte Sensor ist ein Temperatursensor, welcher für dieses Experiment jedoch irrelevant ist.

\subsubsection*{Messung der Unwucht}
Hier kommen dann Bilder rein und wir erklären und diskutieren die Ergebnisse dann, oder?

\subsubsection*{Probleme bei der Detektion}
Nachfolgende Probleme traten bei der Verwendung des Sensors auf:
- Das Auslesen der Sensordaten dauert zu lange. Es ist möglich 2-3 Messergebnisse (Variation hängt von Gewicht des Propellers ab) innerhalb einer Umdrehung (~20ms) des Propellers zu erhalten. Ein Versuch dieses Verhalten zu verbessern bestand darin die Verbindungsgeschwindigkeit zwischen \textit{Arduino} und Sensor zu erhöhen. Die Verbindung zwischen den beiden Geräten findet über die Schnittstelle I\textsuperscript{2}C im normalen Modus statt. Durch Manipulation der \textit{twi.h}, eine Header-Datei der Arduino-Bibliothek \textit{Wire.h}, welcher die Verbindungsparameter für die I\textsuperscript{2}C-Schnittstelle steuert, war es möglich die Verbindung im Fast-Mode zu betreiben. Damit wurde die Anzahl an Messergebnisse pro Umdrehung von 2-3 auf 3-5 Messungen erhöht. Dies ist aber weiterhin zu wenig um eine konkrete Entscheidung über die Lokalisation der Unwucht treffen zu können.
Es existiert eine weitere Möglichkeit den Sensor auf eine höhere Messrate zu bekommen, welche im Datenblatt jedoch nur sehr kurz und knapp erklärt ist. Hierbei ist es möglich das Gyroskop des Sensor abzuschalten und den Beschleunigungssensor mittels eines "Burst-Modus" (multiple reads) auszulesen. Den Sensor in diesen Modus zu setzen und auszulesen war allerdings aufgrund Mangel des technischen Verständnisses und der Datenblatterklärung nicht möglich.
Es kann davon ausgegangen werden, dass die benötigte Zeit für die Messung der Sensoren bei dem eingesetzten Sensor schon intern zu lange dauert. Die Empfehlung (für zukünftige Projekte ?) liegt daher auf dem Einsatz von anderen Sensoren, welche für diesen Einsatzzweck besser geeignet sind. Ein Beispiel hierfür ist der Bosch BMA180. Dies ist ein dreiachsichger Beschleunigungssensor, welcher intern mit einer Messgeschwindigkeit von 1200Hz operieren kann. Damit kommt dieser Sensor auf ein komplett neues Messergebniss alle 417$\mu$s (Quelle: http://irtfweb.ifa.hawaii.edu/~tcs3/jumpman/jumppc/1107-BMA180/BMA180-DataSheet-v2.5.pdf). Hiermit sind bei einer Umdrehungszeit von ~20ms bis zu 48 Messergebnisse möglich, welche wahrscheinlich ausreichend sind für eine Aussage über die Unwuchtlokalisierung. Eine weitere Möglichkeit für den Ersatz des Sensor wäre der Einsatz einer inertialen Messeinheit (englisch \textit{inertial measurement unit}, IMU). Dies ist meist eine räumliche Kombination von mehreren Sensoren, welche auch Einsatz in heutigen Drohnen finden. Ob sich diese für den hier benötigten Einsatz verwenden liesen ist nicht bekannt und müsste weiter recherchiert werden.
