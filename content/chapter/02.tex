%!TEX root = ../../main.tex

\chapter{Theoretische Grundlagen}

\section{Unwucht}

\section{Mikrocontrollerplattform Arduino}
\textit{Arduino} ist eine aus Soft- und Hardware bestehende Plattform, dessen Komponenten nach dem Prinzip von OpenSource durch die Lizenzen \ac{LGPL} oder \ac{GPL} komplett quelloffen gehalten werden.
Die Hardware besteht aus einem E/A-Board mit aufgebrachtem Mikrocontroller und mehreren digitalen, als auch analogen Ein- und Ausgängen.
Beispielhaft ist dies in Abbildung \ref{fig:arduino_uno_schema} anhand eines Arduino UNOs dargestellt.
\begin{figure}[H]
	\centering
	\includegraphics[width=5cm]{images/chapter/02/arduino_uno.png}
	\caption{Schamtische Darstellung eines Arduino UNOs}
	\label{fig:arduino_uno_schema}
\end{figure}
Die Boards stellen eine serielle Schnittstelle, unter anderem auch \ac{USB} zur Verfügung, über welche unter anderem die Programmierung des Mikrocontrollers stattfindet.
Die Software für den Arduino wird in einer C, bzw. C++ ähnlichen Programmiersprache geschrieben.
Für das Entwickeln des Programmes und das Programmiern des Mikrocontrollers wird von dem Hersteller eine Entwicklungsumgebung zur Verfügung gestellt, welche diese Aufgaben vereinfacht.

Ein erfolgreiches Konzept der Arduino-Plattform sind die sogenannten \textit{Shields}.
Diese sind Zusatzleiterplatinen welche unkompliziert auf das Arduino-Board gesteckt werden und so die Funktionsvielfalt des Arduinos erweitert wird.
Ein besonderes Merkmal hierbei ist die Möglichkeit, dass mehrere \textit{Shields} aufeinander gesteckt werden können um so platzsparend mehrere Zusatzleiterplatinen an ein einzelnes Arduino-Board anzuschließen.
\begin{figure}[H]
	\centering
	\includegraphics[width=5cm]{images/chapter/02/arduino_shield.jpg}
	\caption{Fotografie eines Arduino UNOs mit Shield}
	\label{fig:arduino_shield}
\end{figure}
